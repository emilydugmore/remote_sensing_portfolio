% Options for packages loaded elsewhere
% Options for packages loaded elsewhere
\PassOptionsToPackage{unicode}{hyperref}
\PassOptionsToPackage{hyphens}{url}
\PassOptionsToPackage{dvipsnames,svgnames,x11names}{xcolor}
%
\documentclass[
  letterpaper,
  DIV=11,
  numbers=noendperiod]{scrreprt}
\usepackage{xcolor}
\usepackage{amsmath,amssymb}
\setcounter{secnumdepth}{5}
\usepackage{iftex}
\ifPDFTeX
  \usepackage[T1]{fontenc}
  \usepackage[utf8]{inputenc}
  \usepackage{textcomp} % provide euro and other symbols
\else % if luatex or xetex
  \usepackage{unicode-math} % this also loads fontspec
  \defaultfontfeatures{Scale=MatchLowercase}
  \defaultfontfeatures[\rmfamily]{Ligatures=TeX,Scale=1}
\fi
\usepackage{lmodern}
\ifPDFTeX\else
  % xetex/luatex font selection
\fi
% Use upquote if available, for straight quotes in verbatim environments
\IfFileExists{upquote.sty}{\usepackage{upquote}}{}
\IfFileExists{microtype.sty}{% use microtype if available
  \usepackage[]{microtype}
  \UseMicrotypeSet[protrusion]{basicmath} % disable protrusion for tt fonts
}{}
\makeatletter
\@ifundefined{KOMAClassName}{% if non-KOMA class
  \IfFileExists{parskip.sty}{%
    \usepackage{parskip}
  }{% else
    \setlength{\parindent}{0pt}
    \setlength{\parskip}{6pt plus 2pt minus 1pt}}
}{% if KOMA class
  \KOMAoptions{parskip=half}}
\makeatother
% Make \paragraph and \subparagraph free-standing
\makeatletter
\ifx\paragraph\undefined\else
  \let\oldparagraph\paragraph
  \renewcommand{\paragraph}{
    \@ifstar
      \xxxParagraphStar
      \xxxParagraphNoStar
  }
  \newcommand{\xxxParagraphStar}[1]{\oldparagraph*{#1}\mbox{}}
  \newcommand{\xxxParagraphNoStar}[1]{\oldparagraph{#1}\mbox{}}
\fi
\ifx\subparagraph\undefined\else
  \let\oldsubparagraph\subparagraph
  \renewcommand{\subparagraph}{
    \@ifstar
      \xxxSubParagraphStar
      \xxxSubParagraphNoStar
  }
  \newcommand{\xxxSubParagraphStar}[1]{\oldsubparagraph*{#1}\mbox{}}
  \newcommand{\xxxSubParagraphNoStar}[1]{\oldsubparagraph{#1}\mbox{}}
\fi
\makeatother


\usepackage{longtable,booktabs,array}
\usepackage{calc} % for calculating minipage widths
% Correct order of tables after \paragraph or \subparagraph
\usepackage{etoolbox}
\makeatletter
\patchcmd\longtable{\par}{\if@noskipsec\mbox{}\fi\par}{}{}
\makeatother
% Allow footnotes in longtable head/foot
\IfFileExists{footnotehyper.sty}{\usepackage{footnotehyper}}{\usepackage{footnote}}
\makesavenoteenv{longtable}
\usepackage{graphicx}
\makeatletter
\newsavebox\pandoc@box
\newcommand*\pandocbounded[1]{% scales image to fit in text height/width
  \sbox\pandoc@box{#1}%
  \Gscale@div\@tempa{\textheight}{\dimexpr\ht\pandoc@box+\dp\pandoc@box\relax}%
  \Gscale@div\@tempb{\linewidth}{\wd\pandoc@box}%
  \ifdim\@tempb\p@<\@tempa\p@\let\@tempa\@tempb\fi% select the smaller of both
  \ifdim\@tempa\p@<\p@\scalebox{\@tempa}{\usebox\pandoc@box}%
  \else\usebox{\pandoc@box}%
  \fi%
}
% Set default figure placement to htbp
\def\fps@figure{htbp}
\makeatother





\setlength{\emergencystretch}{3em} % prevent overfull lines

\providecommand{\tightlist}{%
  \setlength{\itemsep}{0pt}\setlength{\parskip}{0pt}}



 


\KOMAoption{captions}{tableheading}
\makeatletter
\@ifpackageloaded{bookmark}{}{\usepackage{bookmark}}
\makeatother
\makeatletter
\@ifpackageloaded{caption}{}{\usepackage{caption}}
\AtBeginDocument{%
\ifdefined\contentsname
  \renewcommand*\contentsname{Table of contents}
\else
  \newcommand\contentsname{Table of contents}
\fi
\ifdefined\listfigurename
  \renewcommand*\listfigurename{List of Figures}
\else
  \newcommand\listfigurename{List of Figures}
\fi
\ifdefined\listtablename
  \renewcommand*\listtablename{List of Tables}
\else
  \newcommand\listtablename{List of Tables}
\fi
\ifdefined\figurename
  \renewcommand*\figurename{Figure}
\else
  \newcommand\figurename{Figure}
\fi
\ifdefined\tablename
  \renewcommand*\tablename{Table}
\else
  \newcommand\tablename{Table}
\fi
}
\@ifpackageloaded{float}{}{\usepackage{float}}
\floatstyle{ruled}
\@ifundefined{c@chapter}{\newfloat{codelisting}{h}{lop}}{\newfloat{codelisting}{h}{lop}[chapter]}
\floatname{codelisting}{Listing}
\newcommand*\listoflistings{\listof{codelisting}{List of Listings}}
\makeatother
\makeatletter
\makeatother
\makeatletter
\@ifpackageloaded{caption}{}{\usepackage{caption}}
\@ifpackageloaded{subcaption}{}{\usepackage{subcaption}}
\makeatother
\usepackage{bookmark}
\IfFileExists{xurl.sty}{\usepackage{xurl}}{} % add URL line breaks if available
\urlstyle{same}
\hypersetup{
  pdftitle={rs\_portfolio\_proj},
  pdfauthor={Norah Jones},
  colorlinks=true,
  linkcolor={blue},
  filecolor={Maroon},
  citecolor={Blue},
  urlcolor={Blue},
  pdfcreator={LaTeX via pandoc}}


\title{rs\_portfolio\_proj}
\author{Norah Jones}
\date{2026-01-20}
\begin{document}
\maketitle

\renewcommand*\contentsname{Table of contents}
{
\hypersetup{linkcolor=}
\setcounter{tocdepth}{2}
\tableofcontents
}

\bookmarksetup{startatroot}

\chapter*{About}\label{about}
\addcontentsline{toc}{chapter}{About}

\markboth{About}{About}

Introduction to myself blah blah blah

\bookmarksetup{startatroot}

\chapter*{Introduction to Remote
Sensing}\label{introduction-to-remote-sensing}
\addcontentsline{toc}{chapter}{Introduction to Remote Sensing}

\markboth{Introduction to Remote Sensing}{Introduction to Remote
Sensing}

\section*{What is remote sensing?}\label{what-is-remote-sensing}
\addcontentsline{toc}{section}{What is remote sensing?}

\markright{What is remote sensing?}

Put simply, remote sensing is a method of acquiring information from a
distance through sensors mounted on a platform (e.g., satellites,
planes, drones).

\subsection*{Active vs passive sensors}\label{active-vs-passive-sensors}
\addcontentsline{toc}{subsection}{Active vs passive sensors}

\textbf{Passive sensors} rely on naturally available energy, primarily
sunlight, and do not emit energy themselves. As solar electromagnetic
radiation (EMR) travels through the atmosphere and reflects off the
Earth's surface, it undergoes several interactions, including
absorption, transmission, and scattering. These interactions can
significantly reduce the amount of energy that reaches the sensor.
Consequently, passive sensors are ineffective in low-light conditions
and are unable to penetrate obstacles such as clouds, smoke, or dense
vegetation, as these features scatter or absorb the reflected radiation.

\textbf{Active sensors} emit their own EMR and wait to receive the
reflected energy. The emitted energy is often in the form of long
wavelengths that are able to `pass through' atmospheric obstacles which
have smaller particle sizes (rather than being scattered, absorbed or
reflected).

\subsection*{Spectral Signatures}\label{spectral-signatures}
\addcontentsline{toc}{subsection}{Spectral Signatures}

Spectral signatures show how different materials reflect or absorb
electromagnetic energy across a spectrum of wavelengths on the
electromagnetic spectrum. Each feature on Earth has a unique spectral
signature that is determined by physical and chemical properties and how
it interacts with electromagnetic radiation.

\pandocbounded{\includegraphics[keepaspectratio]{images/spectral_sig.png}}

An important feature in the spectral signature is the \textbf{red edge}
- a sharp increase in reflectance around 700\,nm in vegetation's
spectral signature, which indicates chlorophyll content and plant
health.

\subsection*{Resolutions}\label{resolutions}
\addcontentsline{toc}{subsection}{Resolutions}

The characteristics of remote sensors will determine the level of
accuracy and detail of the information about the Earth's surface.

\subsubsection*{Spectral Resolution}\label{spectral-resolution}
\addcontentsline{toc}{subsubsection}{Spectral Resolution}

Spectral resolution refers to a sensor's ability to distinguish between
different wavelengths of electromagnetic radiation from the received
signal. Each spectral band corresponds to a specific wavelength range,
and averages its information across this range. Wider spectral bands
reflect a lower spectral resolution. A higher number of bands reflects a
higher spectral resolution.

Sometimes there are large gaps of wavelength ranges in the EMS in which
no information is collected, and this is because the atmosphere does not
allow certain wavelengths to pass. Thus, bands are often limited to
atmospheric windows where wavelengths can penetrate.

\begin{figure}[H]

{\centering \pandocbounded{\includegraphics[keepaspectratio]{images/spectral_resolution.png}}

}

\caption{Spectral Resolutions of two satellites. Landsat 1 (MSS) has
four spectral bands within the visible and near IR spectra, while
Landsat 8 has 15 bands spread across the whole electromagnetic
spectrum.}

\end{figure}%

\paragraph*{What happens next?}\label{what-happens-next}
\addcontentsline{toc}{paragraph}{What happens next?}

Looking more closely at the Landsat 1 (MSS) satellite\ldots{}

The information captured by each of the four spectral bands are stored
as a greyscale image, which shows the reflection intensity of the
Earth's surface within each band. By combining the refletance
information from each band, and then comparing this with spectral
signatures, we can distinguish features of the Earth's surface.

\pandocbounded{\includegraphics[keepaspectratio]{images/clipboard-3618918105.png}}

\subsubsection*{Other Resolutions}\label{other-resolutions}
\addcontentsline{toc}{subsubsection}{Other Resolutions}

\textbf{Spatial} = The size of the raster grid per pixel, from 10cm to
several kilometers.

\textbf{Temporal} = The time between revisits of information collection
for a given location.

\textbf{Radiometric} = The ability of a sensor to identify and show
small differences in energy.

*** Something about balancing / trade offs ***

\subsection*{Colour composites}\label{colour-composites}
\addcontentsline{toc}{subsection}{Colour composites}

Colour composites are images created by assigning different spectral
bands to the red, green, and blue (RGB) display bands, and are used to
highlight specific features that are often not visible to the human eye.

\begin{figure}[H]

{\centering \pandocbounded{\includegraphics[keepaspectratio]{index_files/mediabag/Comparison-of-spectr.pbm}}

}

\caption{Comparison of the spectral resolutions of Landsat 7 \& 8 with
Sentinel-2}

\end{figure}%

\textbf{True colour composites} (B4, B3, B2 for Sentinel-2) display the
Earth as we would see it with our eyes.

\textbf{False colour composites} use non-visible bands in the
near-infrared range alongside red and green (B8, B4, B3 for Sentinel-2)
to provide information particularly useful for vegetation analysis. The
choice of these three bands has to do with plants reflecting
near-infrared and green light while absorbing red light (see
\textbf{spectral signatures} for a reminder of this).

Expanding on this a bit more, we can make scatterplots (called a
spectral feature space) of two bands against one another to see how much
of our study area is covered by different features (in this case, dense
vs sparse vegetation and dry vs wet soil).

\begin{figure}[H]

{\centering \pandocbounded{\includegraphics[keepaspectratio]{images/clipboard-913266028.png}}

}

\caption{Source: Remote Sensing 4113}

\end{figure}%

\textbf{Atmospheric penetration composites} (B12, B11, B8A for
Sentinel-2) have no visible bands so that the image penetrate
atmospheric particles and reduces the effect of atmospheric scattering.
The resulting image from this composite shows vegetation as blue and
urban areas as white/grey or cyan/purple.

\pandocbounded{\includegraphics[keepaspectratio]{index_files/mediabag/Fig5_6.jpg}}

\subsubsection*{Tasseled Cap function}\label{tasseled-cap-function}
\addcontentsline{toc}{subsubsection}{Tasseled Cap function}

Although this isn't a true colour composite, and rather acts more as a
spectral index, it's interpretation is somewhat similar.

The tasseled cap function is a spectral transformation that combines
multiple spectral bands into three new composite layers - brightness,
greenness, and wetness (which are assigned to the red, green and blue
bands, respectively). The resulting image is a representation of
different land cover types (red = urban structures, green = vegetation,
blue = water/moisture).

\pandocbounded{\includegraphics[keepaspectratio]{images/clipboard-409740593.png}}

\section*{LandSat \& Sentinel
Applications}\label{landsat-sentinel-applications}
\addcontentsline{toc}{section}{LandSat \& Sentinel Applications}

\markright{LandSat \& Sentinel Applications}

CASA's very own Ollie Ballinger used satellite images (with the help of
complex AI software) from \emph{Copernicus' Sentinel Hub} to identify
and track US Navy vessels operating with disabled transponders in the
Caribbean. His findings appeared in a \emph{New York Times} article,
revealing a larger, more consistent presence of US Navy ships in the
region than officials had admitted. The research ultimately pre-empted
Operation Absolute Resolve in January 2026 - the large-scale US military
strike that led to the capture of Venezuelan President Nicolás Maduro.

\pandocbounded{\includegraphics[keepaspectratio]{images/clipboard-1079227062.png}}

A bit more up my area of expertise (coming from background of ecology
and conservation biology), is a study
(\url{https://www.mdpi.com/2072-4292/16/5/798}) which usedsatellite
imagery from Landsat 5 TM and Landsat 8 OLI to map and monitor the
spread of the invasive plant \emph{Dichrostachys cinerea} (marabú) in
central Cuba over nearly three decades.

Something I thought was particularly interesting about this study is
that it went beyond simply mapping the extent of invasive spread, by
also quantifying the impact that its presence had on specific land
covers. This is an important aspect of invasive species studies (and I
think one which is often skipped), and shows that remote sensing can be
used to inform management strategies and detect areas of priority.

\subsection*{Sentinel-2 vs LandSat 8}\label{sentinel-2-vs-landsat-8}
\addcontentsline{toc}{subsection}{Sentinel-2 vs LandSat 8}

\begin{longtable}[]{@{}
  >{\raggedright\arraybackslash}p{(\linewidth - 4\tabcolsep) * \real{0.3333}}
  >{\raggedright\arraybackslash}p{(\linewidth - 4\tabcolsep) * \real{0.3333}}
  >{\raggedright\arraybackslash}p{(\linewidth - 4\tabcolsep) * \real{0.3333}}@{}}
\toprule\noalign{}
\begin{minipage}[b]{\linewidth}\raggedright
Feature
\end{minipage} & \begin{minipage}[b]{\linewidth}\raggedright
Landsat 8
\end{minipage} & \begin{minipage}[b]{\linewidth}\raggedright
Sentinel-2
\end{minipage} \\
\midrule\noalign{}
\endhead
\bottomrule\noalign{}
\endlastfoot
\textbf{Satellite Constellation} & Single satellite & 2 satellites (2A
\& 2B) \\
\textbf{Spatial Resolution} & 30 m & 10 m, 20 m, 60 m depending on
band \\
\textbf{Panchromatic Band} & 15 m & None \\
\textbf{Thermal Band} & Yes, 100 m (resampled to 30 m) & No \\
\textbf{Red Edge Bands} & No & 3 bands (705, 740, 783 nm)* \\
\textbf{Number of Bands} & 11 & 13 \\
\textbf{Revisit Time} & 16 days & 5 days \\
\textbf{Key Applications} & Land use/land cover mapping, thermal
monitoring, water temperature, urban heat studies & Agriculture,
forestry, water quality, vegetation health, disaster monitoring \\
\end{longtable}

* More bands within the red edge is useful for vegetation application
due to the red edge!

This is a sentence with a {hover note}.

\section*{Reflection}\label{reflection}
\addcontentsline{toc}{section}{Reflection}

\markright{Reflection}

Academically, my use of remotely sensed imagery has been limited to
using true colour composites as a means of contextualizing a study area
and setting the scene in the form of basemaps. Personally, I use Google
Earth extensively to look for different hiking routes and places that
look interesting when I am travelling.

This is the first time that I am going deeper than that. Even just from
the first week it feels like I have learned to much about how remote
sensors collect information, downloading the data and manipulating it to
suite the needs of different applications.

Something that stood out for me was colour composites and spectral
indexes. In my previous degree, Normalised Difference Vegetation Index
(NDVI) was spoken about a lot as a proxy for vegetation health. Finally
understanding how this is calculated using remote sensors (and distinct
properties of the spectral signatures of vegetation) is really
satisfying!

\section*{References}\label{references}
\addcontentsline{toc}{section}{References}

\markright{References}

\url{https://www.nv5geospatialsoftware.com/What-is-Remote-Sensing}

spectral signature image =
\url{https://www.researchgate.net/figure/Spectral-signatures-as-functions-of-wavelength-for-five-typical-surfaces-The-central_fig4_318843407}

\url{https://www.instagram.com/p/DTRSx-6kbXO/}

chrome-extension://efaidnbmnnnibpcajpcglclefindmkaj/https://geofaculty.uwyo.edu/rhowell/classes/remote\_sensing/labs/lab\_11\_2018\_alternate\_vegetation.pdf

\url{https://ltb.itc.utwente.nl/498/concept/81525}




\end{document}
