% Options for packages loaded elsewhere
% Options for packages loaded elsewhere
\PassOptionsToPackage{unicode}{hyperref}
\PassOptionsToPackage{hyphens}{url}
\PassOptionsToPackage{dvipsnames,svgnames,x11names}{xcolor}
%
\documentclass[
  letterpaper,
  DIV=11,
  numbers=noendperiod]{scrreprt}
\usepackage{xcolor}
\usepackage{amsmath,amssymb}
\setcounter{secnumdepth}{5}
\usepackage{iftex}
\ifPDFTeX
  \usepackage[T1]{fontenc}
  \usepackage[utf8]{inputenc}
  \usepackage{textcomp} % provide euro and other symbols
\else % if luatex or xetex
  \usepackage{unicode-math} % this also loads fontspec
  \defaultfontfeatures{Scale=MatchLowercase}
  \defaultfontfeatures[\rmfamily]{Ligatures=TeX,Scale=1}
\fi
\usepackage{lmodern}
\ifPDFTeX\else
  % xetex/luatex font selection
\fi
% Use upquote if available, for straight quotes in verbatim environments
\IfFileExists{upquote.sty}{\usepackage{upquote}}{}
\IfFileExists{microtype.sty}{% use microtype if available
  \usepackage[]{microtype}
  \UseMicrotypeSet[protrusion]{basicmath} % disable protrusion for tt fonts
}{}
\makeatletter
\@ifundefined{KOMAClassName}{% if non-KOMA class
  \IfFileExists{parskip.sty}{%
    \usepackage{parskip}
  }{% else
    \setlength{\parindent}{0pt}
    \setlength{\parskip}{6pt plus 2pt minus 1pt}}
}{% if KOMA class
  \KOMAoptions{parskip=half}}
\makeatother
% Make \paragraph and \subparagraph free-standing
\makeatletter
\ifx\paragraph\undefined\else
  \let\oldparagraph\paragraph
  \renewcommand{\paragraph}{
    \@ifstar
      \xxxParagraphStar
      \xxxParagraphNoStar
  }
  \newcommand{\xxxParagraphStar}[1]{\oldparagraph*{#1}\mbox{}}
  \newcommand{\xxxParagraphNoStar}[1]{\oldparagraph{#1}\mbox{}}
\fi
\ifx\subparagraph\undefined\else
  \let\oldsubparagraph\subparagraph
  \renewcommand{\subparagraph}{
    \@ifstar
      \xxxSubParagraphStar
      \xxxSubParagraphNoStar
  }
  \newcommand{\xxxSubParagraphStar}[1]{\oldsubparagraph*{#1}\mbox{}}
  \newcommand{\xxxSubParagraphNoStar}[1]{\oldsubparagraph{#1}\mbox{}}
\fi
\makeatother


\usepackage{longtable,booktabs,array}
\usepackage{calc} % for calculating minipage widths
% Correct order of tables after \paragraph or \subparagraph
\usepackage{etoolbox}
\makeatletter
\patchcmd\longtable{\par}{\if@noskipsec\mbox{}\fi\par}{}{}
\makeatother
% Allow footnotes in longtable head/foot
\IfFileExists{footnotehyper.sty}{\usepackage{footnotehyper}}{\usepackage{footnote}}
\makesavenoteenv{longtable}
\usepackage{graphicx}
\makeatletter
\newsavebox\pandoc@box
\newcommand*\pandocbounded[1]{% scales image to fit in text height/width
  \sbox\pandoc@box{#1}%
  \Gscale@div\@tempa{\textheight}{\dimexpr\ht\pandoc@box+\dp\pandoc@box\relax}%
  \Gscale@div\@tempb{\linewidth}{\wd\pandoc@box}%
  \ifdim\@tempb\p@<\@tempa\p@\let\@tempa\@tempb\fi% select the smaller of both
  \ifdim\@tempa\p@<\p@\scalebox{\@tempa}{\usebox\pandoc@box}%
  \else\usebox{\pandoc@box}%
  \fi%
}
% Set default figure placement to htbp
\def\fps@figure{htbp}
\makeatother





\setlength{\emergencystretch}{3em} % prevent overfull lines

\providecommand{\tightlist}{%
  \setlength{\itemsep}{0pt}\setlength{\parskip}{0pt}}



 


\KOMAoption{captions}{tableheading}
\makeatletter
\@ifpackageloaded{bookmark}{}{\usepackage{bookmark}}
\makeatother
\makeatletter
\@ifpackageloaded{caption}{}{\usepackage{caption}}
\AtBeginDocument{%
\ifdefined\contentsname
  \renewcommand*\contentsname{Table of contents}
\else
  \newcommand\contentsname{Table of contents}
\fi
\ifdefined\listfigurename
  \renewcommand*\listfigurename{List of Figures}
\else
  \newcommand\listfigurename{List of Figures}
\fi
\ifdefined\listtablename
  \renewcommand*\listtablename{List of Tables}
\else
  \newcommand\listtablename{List of Tables}
\fi
\ifdefined\figurename
  \renewcommand*\figurename{Figure}
\else
  \newcommand\figurename{Figure}
\fi
\ifdefined\tablename
  \renewcommand*\tablename{Table}
\else
  \newcommand\tablename{Table}
\fi
}
\@ifpackageloaded{float}{}{\usepackage{float}}
\floatstyle{ruled}
\@ifundefined{c@chapter}{\newfloat{codelisting}{h}{lop}}{\newfloat{codelisting}{h}{lop}[chapter]}
\floatname{codelisting}{Listing}
\newcommand*\listoflistings{\listof{codelisting}{List of Listings}}
\makeatother
\makeatletter
\makeatother
\makeatletter
\@ifpackageloaded{caption}{}{\usepackage{caption}}
\@ifpackageloaded{subcaption}{}{\usepackage{subcaption}}
\makeatother
\usepackage{bookmark}
\IfFileExists{xurl.sty}{\usepackage{xurl}}{} % add URL line breaks if available
\urlstyle{same}
\hypersetup{
  pdftitle={rs\_portfolio\_proj},
  pdfauthor={Norah Jones},
  colorlinks=true,
  linkcolor={blue},
  filecolor={Maroon},
  citecolor={Blue},
  urlcolor={Blue},
  pdfcreator={LaTeX via pandoc}}


\title{rs\_portfolio\_proj}
\author{Norah Jones}
\date{2026-01-20}
\begin{document}
\maketitle

\renewcommand*\contentsname{Table of contents}
{
\hypersetup{linkcolor=}
\setcounter{tocdepth}{2}
\tableofcontents
}

\bookmarksetup{startatroot}

\chapter*{About}\label{about}
\addcontentsline{toc}{chapter}{About}

\markboth{About}{About}

Introduction to myself blah blah blah

\bookmarksetup{startatroot}

\chapter*{Introduction to Remote
Sensing}\label{introduction-to-remote-sensing}
\addcontentsline{toc}{chapter}{Introduction to Remote Sensing}

\markboth{Introduction to Remote Sensing}{Introduction to Remote
Sensing}

\section*{What is remote sensing?}\label{what-is-remote-sensing}
\addcontentsline{toc}{section}{What is remote sensing?}

\markright{What is remote sensing?}

Put simply, remote sensing is a method of acquiring information from a
distance through sensors mounted on a platform (e.g., satellites,
planes, drones).

\subsection*{Active vs passive sensors}\label{active-vs-passive-sensors}
\addcontentsline{toc}{subsection}{Active vs passive sensors}

\textbf{Passive sensors} rely on naturally available energy, primarily
sunlight, and do not emit energy themselves. As solar electromagnetic
radiation (EMR) travels through the atmosphere and reflects off the
Earth's surface, it undergoes several interactions, including
absorption, transmission, and scattering. These interactions can
significantly reduce the amount of energy that reaches the sensor.
Consequently, passive sensors are ineffective in low-light conditions
and are unable to penetrate obstacles such as clouds, smoke, or dense
vegetation, as these features scatter or absorb the reflected radiation.

\textbf{Active sensors} emit their own EMR and wait to receive the
reflected energy. The emitted energy is often in the form of long
wavelengths that are able to `pass through' atmospheric obstacles which
have smaller particle sizes (rather than being scattered, absorbed or
reflected).

\subsection*{Spectral Signatures}\label{spectral-signatures}
\addcontentsline{toc}{subsection}{Spectral Signatures}

Spectral signatures show how different materials reflect or absorb
electromagnetic energy across a spectrum of wavelengths on the
electromagnetic spectrum. Each feature on Earth has a unique spectral
signature that is determined by physical and chemical properties and how
it interacts with electromagnetic radiation.

\pandocbounded{\includegraphics[keepaspectratio]{images/spectral_sig.png}}

\subsection*{Resolutions}\label{resolutions}
\addcontentsline{toc}{subsection}{Resolutions}

The characteristics of remote sensors will determine the level of
accuracy and detail of the information about the Earth's surface.

\subsubsection*{Spectral Resolution}\label{spectral-resolution}
\addcontentsline{toc}{subsubsection}{Spectral Resolution}

Spectral resolution refers to a sensor's ability to distinguish between
different wavelengths of electromagnetic radiation from the received
signal. Each spectral band corresponds to a specific wavelength range,
and averages its information across this range. Wider spectral bands
reflect a lower spectral resolution. A higher number of bands reflects a
higher spectral resolution.

Sometimes there are large gaps of wavelength ranges in the EMS in which
no information is collected, and this is because the atmosphere does not
allow certain wavelengths to pass. Thus, bands are often limited to
atmospheric windows where wavelengths can penetrate.

\begin{figure}[H]

{\centering \pandocbounded{\includegraphics[keepaspectratio]{images/spectral_resolution.png}}

}

\caption{Spectral Resolutions of two satellites. Landsat 1 (MSS) has
four spectral bands within the visible and near IR spectra, while
Landsat 8 has 15 bands spread across the whole electromagnetic
spectrum.}

\end{figure}%

\paragraph*{What happens next?}\label{what-happens-next}
\addcontentsline{toc}{paragraph}{What happens next?}

Looking more closely at the Landsat 1 (MSS) satellite\ldots{}

The information captured by each of the four spectral bands are stored
as a greyscale image, which shows the reflection intensity of the
Earth's surface within each band. By combining the refletance
information from each band, and then comparing this with spectral
signatures, we can distinguish features of the Earth's surface.

\pandocbounded{\includegraphics[keepaspectratio]{images/clipboard-3618918105.png}}

\subsubsection*{Other Resolutions}\label{other-resolutions}
\addcontentsline{toc}{subsubsection}{Other Resolutions}

\textbf{Spatial} = The size of the raster grid per pixel, from 10cm to
several kilometers.

\textbf{Temporal} = The time between revisits of information collection
for a given location.

\textbf{Radiometric} = The ability of a sensor to identify and show
small differences in energy.

*** Something about balancing / trade offs ***

\subsection*{Colour composites}\label{colour-composites}
\addcontentsline{toc}{subsection}{Colour composites}

\pandocbounded{\includegraphics[keepaspectratio]{index_files/mediabag/Comparison-of-spectr.pbm}}

Colour composites are images created by assigning different spectral
bands to the red, green, and blue (RGB) display bands within our
software, and are used to highlight specific features that are often not
visible to the human eye.

\textbf{True colour composites} (B4, B3, B2 for Sentinel-2) display the
Earth as we would see it with our eyes.

\textbf{False colour composites} use non-visible bands in the
near-infrared range alongside red and green (B8, B4, B3 for Sentinel-2)
to provide information particularly useful for vegetation analysis. The
choice of these three bands has to do with plants reflecting
near-infrared and green light while absorbing red light (see
\textbf{spectral signatures} for a reminder of this).

\textbf{Atmospheric penetration composites} (B12, B11, B8A for
Sentinel-2) have no visible bands so that the image penetrate
atmospheric particles and reduces the effect of atmospheric scattering.
The resulting image from this composite shows vegetation as blue and
urban areas as white/grey or cyan/purple.

\subsection*{Sentinel 2}\label{sentinel-2}
\addcontentsline{toc}{subsection}{Sentinel 2}

Many problems were met with loading data on the Sentinel Application
Platform (SNAP). Usually, the software can be used for basic raster
image processing and analysis.

\subsection*{LandSat vs sentinel}\label{landsat-vs-sentinel}
\addcontentsline{toc}{subsection}{LandSat vs sentinel}

kjhdwudwhdo

Reflection

\begin{itemize}
\tightlist
\item
  snapshot of the real world, but sensors can also be used to track
  movements
\end{itemize}

\section*{References}\label{references}
\addcontentsline{toc}{section}{References}

\markright{References}

\url{https://www.nv5geospatialsoftware.com/What-is-Remote-Sensing}

spectral signature image =
\url{https://www.researchgate.net/figure/Spectral-signatures-as-functions-of-wavelength-for-five-typical-surfaces-The-central_fig4_318843407}




\end{document}
